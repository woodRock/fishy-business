\documentclass{article}
\usepackage{spconf,amsmath,amssymb,graphicx,url,float}

% Hide red underlines for URLs. 
\usepackage[hidelinks]{hyperref}

% Stack two figures on top of each other.
\usepackage{subcaption}

% Make caption text font smaller.
\usepackage{caption}
\captionsetup[figure]{font=small,labelfont=small}


\graphicspath{{assets/}}

% Template for ICASSP-2020 paper; to be used with:
%          spconf.sty  - ICASSP/ICIP LaTeX style file, and
%          IEEEbib.bst - IEEE bibliography style file.
% --------------------------------------------------------------------------

% Resources
% ---------
% - FGR instructions https://www.wgtn.ac.nz/fgr/current-phd/provisional-to-full-registration/the-proposal 
% 
% 

\def\x{{\mathbf x}}
\def\L{{\cal L}}
\newcommand{\R}{\mathbb{R}}

% Title.
% ------
\title{Proposal}

% Single address.
% ---------------
\name{
  Jesse Wood, Bing-Xue, Mengjie Zhang, Bach Hoai Nguyen, Daniel Killeen
 \thanks{Thanks to New Zealand Plant \& Food Research for datasets, funding and expertise.}
} 

\address{
  Victoria University \\ 
  Engineering and Computer Science \\ 
  Kelburn, Wellington, New Zealand
}

\begin{document}

\maketitle
%
\begin{abstract}
  % The abstract should appear at the top of the left-hand column of text, about
  % 0.5 inches (12 mm) below the title area and no more than 3.125 inches (80 mm) in
  % length.  Leave a 0.5 inch (12 mm) space between the end of the abstract and the
  % beginning of the main text.  The abstract should contain about 100 to 150
  % words, and should be identical to the abstract text submitted electronically
  % along with the paper cover sheet.  All manuscripts must be in English, printed
  % in black ink.

  Submitted in partial fulfilment of the PhD in Artificial Intelligence.

\end{abstract}
%
\begin{keywords}
  Feature Selection, Gas Chromatography, Support Vector Machines, Visualisation
\end{keywords}
%

\section{INTRODUCTION}
\label{sec:introduction}

\begin{itemize}
  \item Scope - place the problem in the world. 
  \item Specifics to New Zealand, sustainability. 
  \item Fish processing - automation, quality control, contaimination. 
  \item Current state-of-the-art
  \begin{itemize}
    \item GC-MS, manual, time consuming, expensive, destructive, instrumental drift. 
  \end{itemize}
\end{itemize}

\section{Literature}
\label{sec:literature} 

\begin{itemize}
  \item Mass spectrometry \cite{eder1995gas}
  \item REIMS 
  \item Classification 
  \item Feature Selection 
  \item Interpretable ML 
  \item Genetic Programming 
  \item Transfer Learning 
\end{itemize}

\section{Preliminary Work}
\label{sec:preliminary}

\begin{itemize}
  \item Automated Fish Classification on GC-MS data. 
  \item Genetic Programming (GP) for GC-MS data 
  \begin{itemize}
    \item Single-Tree GP 
    \item Multi-tree GP 
  \end{itemize}
  \item REIMS exploratory data analysis 
\end{itemize}

\section{Contributions}
\label{sec:contributions}

\begin{itemize}
  \item Each research question applies to the Hoki and Jack Mackeral datasets. 
  \item For each dataset, hoki and mackeral. 
\end{itemize}

These are the research questions from Plant and Food Research. 

\begin{itemize}
  \item Can REIMS data be used to classify different species tissues? What variables are responsible?
  \begin{itemize}
    \item Classification 
    \item Feature Importance - Interpretable 
  \end{itemize}
  \item Can REIMS data detect mixed-species contaminiation in fish tissues? At what concentration? What varaibles are responsible? 
  \begin{itemize}
    \item Classification 
    \item Regression 
    \item Feature importance - Interpretable
  \end{itemize}
  \item Can REIMS data detect mineral oil contamination in fish? At what concentration? What variables are responsible?
  \begin{itemize}
    \item Classification 
    \item Regression 
    \item Feature importance - Interpretable 
  \end{itemize}
  \item Can REIMS data be used to distinguish between different fish individuals? What variables are responsible?
  \begin{itemize}
    \item Identification 
    \item Feature imporance - Interpretable 
  \end{itemize}
\end{itemize}

\section{Milestones}
\label{sec:milestones}

\begin{itemize}
  \item Literature Review 
  \item EDA 
  \item Preprocessing 
  \item Classification 
  \item Cross-species Contaminiation 
  \item Mineral-oil Contaminiation  
  \item Individual Identification 
  \item Auto ML 
  \item Thesis 
\end{itemize}

\section{Thesis Outline}
\label{sec:outline}

\begin{enumerate}
  \item Introduction 
  \item Background 
  \begin{itemize}
    \item Mass Spectrometry 
    \item REIMS 
    \item Classifcation / Regression 
    \item Interpretable ML
  \end{itemize}
  \item Preparations
  \begin{itemize}
    \item Exploratory Data Analysis 
    \item Preprocessing 
  \end{itemize}
  \item Applications 
  \begin{itemize}
    \item Classification 
    \item Contaminiation Detection 
    \item Individual Identification 
    \item Auto ML  
  \end{itemize} 
  \item Discussion 
  \item Conclusion
\end{enumerate}

\section{Resources}
\label{sec:resources}

\begin{itemize}
  \item Hardware 
  \begin{itemize}
    \item ECS Grid Compute 
    \item Rapoi 
    \item Niwa HPC - via Auckland University
  \end{itemize}
  \item Software 
  \begin{itemize}
    \item Repository - Github 
    \item Project Management - Github Projects 
    \item Programming language - Python
    \item Documentation - Read the Docs  
  \end{itemize}
  \item Experience 
  \begin{itemize}
    \item Field-trip to Callaghan Innovation to see REIMS 
    \item Field-trip to NZ Plant and Food Research (if necessary for future datasets). 
  \end{itemize}
\end{itemize}

\section{Appendix}
\label{sec:appendix}


% This paper has demonstrated the performance of the neural network with non-linear activation functions on a non-linear objective function. ReLU \cite{fukushima1982neocognitron} has shown to be effective, which hints towards its wide adoption in the field of deep learning.

% \subsection{Relation to Prior Work}
% \label{sec:prior}

% The text of the paper should contain discussions on how the paper's
% contributions are related to prior work in the field. It is important
% to put new work in context, to give credit to foundational work, and
% to provide details associated with the previous work that has appeared
% in the literature. This discussion may be a separate, numbered section
% or it may appear elsewhere in the body of the manuscript, but it must
% be present.

% You should differentiate what is new and how your work expands on
% or takes a different path from the prior studies. An example might
% read something to the effect: "The work presented here has focused
% on the formulation of the ABC algorithm, which takes advantage of
% non-uniform time-frequency domain analysis of data. The work by
% Smith and Cohen \cite{goodfellow2016deep} considers only fixed time-domain analysis and
% the work by Jones et al \cite{goodfellow2016deep} takes a different approach based on
% fixed frequency partitioning. While the present study is related
% to recent approaches in time-frequency analysis [3-5], it capitalizes
% on a new feature space, which was not considered in these earlier
% studies."

% \vfill\pagebreak

% \section{REFERENCES}
% \label{sec:refs}

% List and number all bibliographical references at the end of the
% paper. The references can be numbered in alphabetic order or 
% order of appearance in the document. When referring to them in
% the text, type the corresponding reference number in square
% brackets as shown at the end of this sentence \cite{goodfellow2016deep}. An
% additional final page (the fifth page, in most cases) is
% allowed, but must contain only references to the prior
% literature.

% References should be produced using the bibtex program from suitable
% BiBTeX files (here: strings, refs, manuals). The IEEEbib.bst bibliography
% style file from IEEE produces unsorted bibliography list.
% -------------------------------------------------------------------------
\bibliographystyle{IEEEbib}
\bibliography{strings,refs}

\end{document}
