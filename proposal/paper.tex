\documentclass{article}
\usepackage{spconf,amsmath,amssymb,graphicx,url,float}

% Hide red underlines for URLs. 
\usepackage[hidelinks]{hyperref}

% Stack two figures on top of each other.
\usepackage{subcaption}

% Make caption text font smaller.
\usepackage{caption}
\captionsetup[figure]{font=small,labelfont=small}


\graphicspath{{assets/}}

% Template for ICASSP-2020 paper; to be used with:
%          spconf.sty  - ICASSP/ICIP LaTeX style file, and
%          IEEEbib.bst - IEEE bibliography style file.
% --------------------------------------------------------------------------

\def\x{{\mathbf x}}
\def\L{{\cal L}}
\newcommand{\R}{\mathbb{R}}

% Title.
% ------
\title{Proposal}

% Single address.
% ---------------
\name{
  Jesse Wood, Bing-Xue, Mengjie Zhang, Bach Hoai Nguyen, Daniel Killeen
 \thanks{Thanks to New Zealand Plant \& Food Research for datasets, funding and expertise.}
} 

\address{
  Victoria University \\ 
  Engineering and Computer Science \\ 
  Kelburn, Wellington, New Zealand
}

\begin{document}

\maketitle
%
\begin{abstract}
  % The abstract should appear at the top of the left-hand column of text, about
  % 0.5 inches (12 mm) below the title area and no more than 3.125 inches (80 mm) in
  % length.  Leave a 0.5 inch (12 mm) space between the end of the abstract and the
  % beginning of the main text.  The abstract should contain about 100 to 150
  % words, and should be identical to the abstract text submitted electronically
  % along with the paper cover sheet.  All manuscripts must be in English, printed
  % in black ink.

  Submitted in partial fulfilment of the PhD in Artificial Intelligence.

\end{abstract}
%
\begin{keywords}
  Feature Selection, Gas Chromatography, Support Vector Machines, Visualisation
\end{keywords}
%

\section{INTRODUCTION}
\label{sec:introduction}

\cite{bi2020gc}

\section{BACKGROUND}
\label{sec:background}

\section{RELATED WORKS}
\label{sec:related-works}

\section{RESULTS \& DISCUSSIONS}
\label{sec:results}

\section{CONCLUSION}
\label{sec:conclusion}

\section{Appendix}
\label{sec:appendix}

% This paper has demonstrated the performance of the neural network with non-linear activation functions on a non-linear objective function. ReLU \cite{fukushima1982neocognitron} has shown to be effective, which hints towards its wide adoption in the field of deep learning.

% \subsection{Relation to Prior Work}
% \label{sec:prior}

% The text of the paper should contain discussions on how the paper's
% contributions are related to prior work in the field. It is important
% to put new work in context, to give credit to foundational work, and
% to provide details associated with the previous work that has appeared
% in the literature. This discussion may be a separate, numbered section
% or it may appear elsewhere in the body of the manuscript, but it must
% be present.

% You should differentiate what is new and how your work expands on
% or takes a different path from the prior studies. An example might
% read something to the effect: "The work presented here has focused
% on the formulation of the ABC algorithm, which takes advantage of
% non-uniform time-frequency domain analysis of data. The work by
% Smith and Cohen \cite{goodfellow2016deep} considers only fixed time-domain analysis and
% the work by Jones et al \cite{goodfellow2016deep} takes a different approach based on
% fixed frequency partitioning. While the present study is related
% to recent approaches in time-frequency analysis [3-5], it capitalizes
% on a new feature space, which was not considered in these earlier
% studies."

% \vfill\pagebreak

% \section{REFERENCES}
% \label{sec:refs}

% List and number all bibliographical references at the end of the
% paper. The references can be numbered in alphabetic order or 
% order of appearance in the document. When referring to them in
% the text, type the corresponding reference number in square
% brackets as shown at the end of this sentence \cite{goodfellow2016deep}. An
% additional final page (the fifth page, in most cases) is
% allowed, but must contain only references to the prior
% literature.

% References should be produced using the bibtex program from suitable
% BiBTeX files (here: strings, refs, manuals). The IEEEbib.bst bibliography
% style file from IEEE produces unsorted bibliography list.
% -------------------------------------------------------------------------
\bibliographystyle{IEEEbib}
\bibliography{strings,refs}

\end{document}
