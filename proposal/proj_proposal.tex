%% $RCSfile: proj_proposal.tex,v $
%% $Revision: 1.4 $
%% $Date: 2017/10/06 02:55:50 $
%% $Author: kevin $

\documentclass[11pt, a4paper, twoside, openright]{report}

\usepackage{float} % lets you have non-floating floats

\usepackage{url} % for typesetting urls

%  We don't want figures to float so we define
%
\newfloat{fig}{thp}{lof}[chapter]
\floatname{fig}{Figure}

%% These are standard LaTeX definitions for the document
%%
\title{Rapid detection and identification of contamination within fish products using rapid ionisation evaporative mass spectrometry}
\author{Jesse Wood}

%% This file can be used for creating a wide range of reports
%%  across various Schools
%%
%% Set up some things, mostly for the front page, for your specific document
%
% Current options are:
% [ecs|msor|sms]          Which school you are in.
%                         (msor option retained for reproducing old data)
% [bschonscomp|mcompsci]  Which degree you are doing
%                          You can also specify any other degree by name
%                          (see below)
% [font|image]            Use a font or an image for the VUW logo
%                          The font option will only work on ECS systems
%
\usepackage[font,ecs]{vuwproject} 

% You should specifiy your supervisor here with
%     \supervisor{Firstname Lastname}
% use \supervisors if there are more than one supervisor

\supervisors{Bach Hoai Nguyen, Bing Xue, Mengjie Zhang}

% Unless you've used the bschonscomp or mcompsci
%  options above use
%   \otherdegree{OTHER DEGREE OR DIPLOMA NAME}
% here to specify degree

\otherdegree{Doctorate of Philosophy - Artificial Intelligence}

% Comment this out if you want the date printed.
\date{}

\begin{document}

% Make the page numbering roman, until after the contents, etc.
\frontmatter

%%%%%%%%%%%%%%%%%%%%%%%%%%%%%%%%%%%%%%%%%%%%%%%%%%%%%%%

\begin{abstract}
  This document gives some ideas about how to write a project
  proposal, and provides a template for a proposal. You should discuss
  your proposal with your supervisor.
\end{abstract}

%%%%%%%%%%%%%%%%%%%%%%%%%%%%%%%%%%%%%%%%%%%%%%%%%%%%%%%

\maketitle

%\tableofcontents

% we want a list of the figures we defined
%\listof{fig}{Figures}

%%%%%%%%%%%%%%%%%%%%%%%%%%%%%%%%%%%%%%%%%%%%%%%%%%%%%%%

\mainmatter

%%%%%%%%%%%%%%%%%%%%%%%%%%%%%%%%%%%%%%%%%%%%%%%%%%%%%%%

\section*{1. Introduction}

% In this section you should include a very brief introduction to the
% problem to the problem and the project.

% Your project proposal should cover the following points:

% \begin{itemize}
% \item the engineering problem that you are going to solve;
% \item how you plan to solve your problem;
% \item how you intend to evaluate your solution; and
% \item any resource requirements for your project such as software,
%   hardware or other resources that will be needed in the course of the
%   project.
% \end{itemize}

% Your proposal should be not more than than 3 pages long.

\begin{itemize}
  \item Scope - place the problem in the world. 
  \item Specifics to New Zealand, sustainability. 
  \item Fish processing - automation, quality control, contaimination. 
  \item Current state-of-the-art
  \begin{itemize}
    \item GC-MS, manual, time consuming, expensive, destructive, instrumental drift. 
  \end{itemize}
\end{itemize}

\section*{2. Literature Review}

% In this section you should give a brief description of the problem
% itself. You want to briefly explain the problem, why it is important
% to solve the problem and define your project aims. After reading this
% section, the reader should understand why it is a problem, believe
% that it is important to solve and have a clear idea of the aims of
% your project.

% When describing the aims of the project, you should avoid vague,
% unmeasurable words like `analyse', `investigate', `describe', and use
% specific, measurable words like `implement', `demonstrate', `show',
% `prove'.

% For example:

% \begin{itemize}
% \item[\bf Good] The aim of this project is to implement and evaluate a
%   management system for network switches;
% \end{itemize}
% is much better than:
% \begin{itemize}
% \item[\bf Bad] The aim of this project is to investigate management
%   systems for network switches.
% \end{itemize}

% In the second case there is no idea of how much work is involved, and
% you will never know whether you have finished. You and your supervisor
% (and the markers of your project) may have very different ideas about
% what such an `investigation' involves. Of course, it is possible that
% the task you set yourself is not achievable, but if you are clear from
% the outset this is less likely, and will more easily be corrected.

\begin{itemize}
  \item Mass spectrometry \cite{eder1995gas}
  \item REIMS 
  \item Classification 
  \item Feature Selection 
  \item Interpretable ML 
  \item Genetic Programming 
  \item Transfer Learning 
\end{itemize}

\section*{3. Prelimary Work}

\begin{itemize}
  \item Automated Fish Classification on GC-MS data. 
  \item CNN for Fish classification on GC-MS data. 
  \item Genetic Programming (GP) for GC-MS data 
  \begin{itemize}
    \item Single-Tree GP 
    \item Multi-tree GP 
  \end{itemize}
  \item REIMS exploratory data analysis 
\end{itemize}


\section*{4. Contributions}

\begin{itemize}
  \item Each research question applies to the REIMS and Data Infusion . 
  \item For each dataset, hoki and mackeral. 
\end{itemize}

These are the research questions from Plant and Food Research. 

\begin{itemize}
  \item Can REIMS data be used to classify different species tissues? What variables are responsible?
  \begin{itemize}
    \item Classification 
    \item Feature Importance - Interpretable 
  \end{itemize}
  \item Can REIMS data detect mixed-species contaminiation in fish tissues? At what concentration? What varaibles are responsible? 
  \begin{itemize}
    \item Classification 
    \item Regression 
    \item Feature importance - Interpretable
  \end{itemize}
  \item Can REIMS data detect mineral oil contamination in fish? At what concentration? What variables are responsible?
  \begin{itemize}
    \item Classification 
    \item Regression 
    \item Feature importance - Interpretable 
  \end{itemize}
  \item Can REIMS data be used to distinguish between different fish individuals? What variables are responsible?
  \begin{itemize}
    \item Identification 
    \item Feature imporance - Interpretable 
  \end{itemize}
\end{itemize}

\section*{5. Milestones}

\begin{itemize}
  \item Literature Review 
  \item EDA 
  \item Preprocessing 
  \item Classification 
  \item Cross-species Contaminiation 
  \item Mineral-oil Contaminiation  
  \item Individual Identification 
  \item Auto ML 
  \item Thesis 
\end{itemize}

\section*{6. Thesis Outline}

\begin{enumerate}
  \item Introduction 
  \item Background 
  \begin{itemize}
    \item Mass Spectrometry 
    \item REIMS 
    \item Classifcation / Regression 
    \item Interpretable ML
  \end{itemize}
  \item Preparations
  \begin{itemize}
    \item Exploratory Data Analysis 
    \item Preprocessing 
  \end{itemize}
  \item Applications 
  \begin{itemize}
    \item Classification 
    \item Contaminiation Detection 
    \item Individual Identification 
    \item Auto ML  
  \end{itemize} 
  \item Discussion 
  \item Conclusion
\end{enumerate}

\section*{7. Resources}

In this section you will detail any resource requirements such as
hardware, software or access to subjects.

\begin{itemize}
  \item Hardware 
  \begin{itemize}
    \item ECS Grid Compute 
    \item Rapoi 
    \item Niwa HPC - via Auckland University
  \end{itemize}
  \item Software 
  \begin{itemize}
    \item Repository - Github 
    \item Project Management - Github Projects 
    \item Programming language - Python
    \item Documentation - Read the Docs  
  \end{itemize}
  \item Experience 
  \begin{itemize}
    \item Field-trip to Callaghan Innovation to see REIMS 
    \item Field-trip to NZ Plant and Food Research (if necessary for future datasets). 
  \end{itemize}
\end{itemize}

%%%%%%%%%%%%%%%%%%%%%%%%%%%%%%%%%%%%%%%%%%%%%%%%%%%%%%%
\backmatter
%%%%%%%%%%%%%%%%%%%%%%%%%%%%%%%%%%%%%%%%%%%%%%%%%%%%%%%

%\bibliographystyle{ieeetr}
\bibliographystyle{acm}
\bibliography{sample}
\end{document}
